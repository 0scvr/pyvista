%% Generated by Sphinx.
\def\sphinxdocclass{report}
\documentclass[letterpaper,10pt,english]{sphinxmanual}
\ifdefined\pdfpxdimen
   \let\sphinxpxdimen\pdfpxdimen\else\newdimen\sphinxpxdimen
\fi \sphinxpxdimen=.75bp\relax

\usepackage[utf8]{inputenc}
\ifdefined\DeclareUnicodeCharacter
 \ifdefined\DeclareUnicodeCharacterAsOptional\else
  \DeclareUnicodeCharacter{00A0}{\nobreakspace}
\fi\fi
\usepackage{cmap}
\usepackage[T1]{fontenc}
\usepackage{amsmath,amssymb,amstext}
\usepackage{babel}
\usepackage{times}
\usepackage[Bjarne]{fncychap}
\usepackage[dontkeepoldnames]{sphinx}

\usepackage{geometry}

% Include hyperref last.
\usepackage{hyperref}
% Fix anchor placement for figures with captions.
\usepackage{hypcap}% it must be loaded after hyperref.
% Set up styles of URL: it should be placed after hyperref.
\urlstyle{same}

\addto\captionsenglish{\renewcommand{\figurename}{Fig.}}
\addto\captionsenglish{\renewcommand{\tablename}{Table}}
\addto\captionsenglish{\renewcommand{\literalblockname}{Listing}}

\addto\extrasenglish{\def\pageautorefname{page}}

\setcounter{tocdepth}{1}



\title{vtkInterface Documentation}
\date{May 31, 2017}
\release{}
\author{Alex Kaszynski}
\newcommand{\sphinxlogo}{\vbox{}}
\renewcommand{\releasename}{Release}
\makeindex

\begin{document}

\maketitle
\sphinxtableofcontents
\phantomsection\label{\detokenize{index::doc}}


vtkInterface is a VTK helper module that takes a different approach on interfacing with VTK through numpy and direct array access.  This module also simplifies mesh creation and viewing by adding functionality to existing VTK objects.

This moudle is suited both for creating engineering plots for presentations and research papers as well as being a supporting module for other mesh heavy software.


\chapter{Installation}
\label{\detokenize{index:vtkinterface-overview}}\label{\detokenize{index:installation}}
If you have a working copy of vtk, installation is simply:

\begin{sphinxVerbatim}[commandchars=\\\{\}]
\PYG{n}{pip} \PYG{n}{install} \PYG{n}{vtkInterface}
\end{sphinxVerbatim}

You can also visit \sphinxhref{http://pypi.python.org/pypi/vtkInterface}{PyPi} or \sphinxhref{https://github.com/akaszynski/vtkInterface}{GitHub} to download the source.  See the {\hyperref[\detokenize{installation:install-ref}]{\sphinxcrossref{\DUrole{std,std-ref}{Installation}}}} for more details.


\chapter{Quick Examples}
\label{\detokenize{index:quick-examples}}

\section{Loading and Plotting a Mesh from File}
\label{\detokenize{index:loading-and-plotting-a-mesh-from-file}}
Loading a mesh is trivial:

\begin{sphinxVerbatim}[commandchars=\\\{\}]
\PYG{k+kn}{import} \PYG{n+nn}{vtkInterface}
\PYG{n}{mesh} \PYG{o}{=} \PYG{n}{vtkInterface}\PYG{o}{.}\PYG{n}{LoadMesh}\PYG{p}{(}\PYG{l+s+s1}{\PYGZsq{}}\PYG{l+s+s1}{airplane.ply}\PYG{l+s+s1}{\PYGZsq{}}\PYG{p}{)}
\PYG{n}{mesh}\PYG{o}{.}\PYG{n}{Plot}\PYG{p}{(}\PYG{n}{color}\PYG{o}{=}\PYG{l+s+s1}{\PYGZsq{}}\PYG{l+s+s1}{orange}\PYG{l+s+s1}{\PYGZsq{}}\PYG{p}{)}
\end{sphinxVerbatim}

\noindent\sphinxincludegraphics{{airplane}.png}

In fact, the code to generate the previous screenshot was created with:

\begin{sphinxVerbatim}[commandchars=\\\{\}]
\PYG{n}{mesh}\PYG{o}{.}\PYG{n}{Plot}\PYG{p}{(}\PYG{n}{screenshot}\PYG{o}{=}\PYG{l+s+s1}{\PYGZsq{}}\PYG{l+s+s1}{airplane.png}\PYG{l+s+s1}{\PYGZsq{}}\PYG{p}{,} \PYG{n}{color}\PYG{o}{=}\PYG{l+s+s1}{\PYGZsq{}}\PYG{l+s+s1}{orange}\PYG{l+s+s1}{\PYGZsq{}}\PYG{p}{)}
\end{sphinxVerbatim}

The points and faces from the mesh are directly accessible as a numpy array:

\begin{sphinxVerbatim}[commandchars=\\\{\}]
\PYG{n+nb}{print} \PYG{n}{mesh}\PYG{o}{.}\PYG{n}{GetNumpyPoints}\PYG{p}{(}\PYG{p}{)}

\PYG{c+c1}{\PYGZsh{}[[ 896.99401855   48.76010132   82.26560211]}
\PYG{c+c1}{\PYGZsh{} [ 906.59301758   48.76010132   80.74520111]}
\PYG{c+c1}{\PYGZsh{} [ 907.53900146   55.49020004   83.65809631]}
\PYG{c+c1}{\PYGZsh{} ...,}
\PYG{c+c1}{\PYGZsh{} [ 806.66497803  627.36297607    5.11482   ]}
\PYG{c+c1}{\PYGZsh{} [ 806.66497803  654.43200684    7.51997995]}
\PYG{c+c1}{\PYGZsh{} [ 806.66497803  681.5369873     9.48744011]]}

\PYG{n+nb}{print} \PYG{n}{mesh}\PYG{o}{.}\PYG{n}{GetNumpyFaces}\PYG{p}{(}\PYG{p}{)}

\PYG{c+c1}{\PYGZsh{}[[   0    1    2]}
\PYG{c+c1}{\PYGZsh{} [   0    2    3]}
\PYG{c+c1}{\PYGZsh{} [   4    5    1]}
\PYG{c+c1}{\PYGZsh{} ...,}
\PYG{c+c1}{\PYGZsh{} [1324 1333 1323]}
\PYG{c+c1}{\PYGZsh{} [1325 1216 1334]}
\PYG{c+c1}{\PYGZsh{} [1325 1334 1324]]}
\end{sphinxVerbatim}


\section{Creating a Structured Surface}
\label{\detokenize{index:creating-a-structured-surface}}
This example creates a simple surface grid and plots the resulting grid and its curvature:

\begin{sphinxVerbatim}[commandchars=\\\{\}]
\PYG{k+kn}{import} \PYG{n+nn}{vtkInterface}

\PYG{c+c1}{\PYGZsh{} Make data}
\PYG{k+kn}{import} \PYG{n+nn}{numpy} \PYG{k}{as} \PYG{n+nn}{np}
\PYG{n}{X} \PYG{o}{=} \PYG{n}{np}\PYG{o}{.}\PYG{n}{arange}\PYG{p}{(}\PYG{o}{\PYGZhy{}}\PYG{l+m+mi}{10}\PYG{p}{,} \PYG{l+m+mi}{10}\PYG{p}{,} \PYG{l+m+mf}{0.25}\PYG{p}{)}
\PYG{n}{Y} \PYG{o}{=} \PYG{n}{np}\PYG{o}{.}\PYG{n}{arange}\PYG{p}{(}\PYG{o}{\PYGZhy{}}\PYG{l+m+mi}{10}\PYG{p}{,} \PYG{l+m+mi}{10}\PYG{p}{,} \PYG{l+m+mf}{0.25}\PYG{p}{)}
\PYG{n}{X}\PYG{p}{,} \PYG{n}{Y} \PYG{o}{=} \PYG{n}{np}\PYG{o}{.}\PYG{n}{meshgrid}\PYG{p}{(}\PYG{n}{X}\PYG{p}{,} \PYG{n}{Y}\PYG{p}{)}
\PYG{n}{R} \PYG{o}{=} \PYG{n}{np}\PYG{o}{.}\PYG{n}{sqrt}\PYG{p}{(}\PYG{n}{X}\PYG{o}{*}\PYG{o}{*}\PYG{l+m+mi}{2} \PYG{o}{+} \PYG{n}{Y}\PYG{o}{*}\PYG{o}{*}\PYG{l+m+mi}{2}\PYG{p}{)}
\PYG{n}{Z} \PYG{o}{=} \PYG{n}{np}\PYG{o}{.}\PYG{n}{sin}\PYG{p}{(}\PYG{n}{R}\PYG{p}{)}

\PYG{c+c1}{\PYGZsh{} Create and plot structured grid}
\PYG{n}{sgrid} \PYG{o}{=} \PYG{n}{vtkInterface}\PYG{o}{.}\PYG{n}{GenStructSurf}\PYG{p}{(}\PYG{n}{X}\PYG{p}{,} \PYG{n}{Y}\PYG{p}{,} \PYG{n}{Z}\PYG{p}{)}
\PYG{n}{sgrid}\PYG{o}{.}\PYG{n}{Plot}\PYG{p}{(}\PYG{p}{)}

\PYG{c+c1}{\PYGZsh{} Plot mean curvature as well}
\PYG{n}{surf}\PYG{o}{.}\PYG{n}{PlotCurvature}\PYG{p}{(}\PYG{p}{)}
\end{sphinxVerbatim}

\noindent\sphinxincludegraphics{{curvature}.png}

Generating a structured grid is a one liner in this module, and the points from the resulting surface are also a numpy array:

\begin{sphinxVerbatim}[commandchars=\\\{\}]
\PYG{n}{surf}\PYG{o}{.}\PYG{n}{GetNumpyPoints}\PYG{p}{(}\PYG{p}{)}

\PYG{c+c1}{\PYGZsh{}[[\PYGZhy{}10.         \PYGZhy{}10.           0.99998766]}
\PYG{c+c1}{\PYGZsh{} [ \PYGZhy{}9.75       \PYGZhy{}10.           0.98546793]}
\PYG{c+c1}{\PYGZsh{} [ \PYGZhy{}9.5        \PYGZhy{}10.           0.9413954 ]}
\PYG{c+c1}{\PYGZsh{} ...,}
\PYG{c+c1}{\PYGZsh{} [  9.25         9.75         0.76645876]}
\PYG{c+c1}{\PYGZsh{} [  9.5          9.75         0.86571785]}
\PYG{c+c1}{\PYGZsh{} [  9.75         9.75         0.93985707]]}
\end{sphinxVerbatim}


\section{Creating a GIF Movie}
\label{\detokenize{index:creating-a-gif-movie}}
This example shows the versatility of the plotting object by generating a moving gif:

\begin{sphinxVerbatim}[commandchars=\\\{\}]
\PYG{k+kn}{import} \PYG{n+nn}{vtkInterface}
\PYG{k+kn}{import} \PYG{n+nn}{numpy} \PYG{k}{as} \PYG{n+nn}{np}

\PYG{c+c1}{\PYGZsh{} Make data}
\PYG{n}{X} \PYG{o}{=} \PYG{n}{np}\PYG{o}{.}\PYG{n}{arange}\PYG{p}{(}\PYG{o}{\PYGZhy{}}\PYG{l+m+mi}{10}\PYG{p}{,} \PYG{l+m+mi}{10}\PYG{p}{,} \PYG{l+m+mf}{0.25}\PYG{p}{)}
\PYG{n}{Y} \PYG{o}{=} \PYG{n}{np}\PYG{o}{.}\PYG{n}{arange}\PYG{p}{(}\PYG{o}{\PYGZhy{}}\PYG{l+m+mi}{10}\PYG{p}{,} \PYG{l+m+mi}{10}\PYG{p}{,} \PYG{l+m+mf}{0.25}\PYG{p}{)}
\PYG{n}{X}\PYG{p}{,} \PYG{n}{Y} \PYG{o}{=} \PYG{n}{np}\PYG{o}{.}\PYG{n}{meshgrid}\PYG{p}{(}\PYG{n}{X}\PYG{p}{,} \PYG{n}{Y}\PYG{p}{)}
\PYG{n}{R} \PYG{o}{=} \PYG{n}{np}\PYG{o}{.}\PYG{n}{sqrt}\PYG{p}{(}\PYG{n}{X}\PYG{o}{*}\PYG{o}{*}\PYG{l+m+mi}{2} \PYG{o}{+} \PYG{n}{Y}\PYG{o}{*}\PYG{o}{*}\PYG{l+m+mi}{2}\PYG{p}{)}
\PYG{n}{Z} \PYG{o}{=} \PYG{n}{np}\PYG{o}{.}\PYG{n}{sin}\PYG{p}{(}\PYG{n}{R}\PYG{p}{)}

\PYG{c+c1}{\PYGZsh{} Create and structured surface}
\PYG{n}{sgrid} \PYG{o}{=} \PYG{n}{vtkInterface}\PYG{o}{.}\PYG{n}{GenStructSurf}\PYG{p}{(}\PYG{n}{X}\PYG{p}{,} \PYG{n}{Y}\PYG{p}{,} \PYG{n}{Z}\PYG{p}{)}

\PYG{c+c1}{\PYGZsh{} Make deep copy of points}
\PYG{n}{pts} \PYG{o}{=} \PYG{n}{sgrid}\PYG{o}{.}\PYG{n}{GetNumpyPoints}\PYG{p}{(}\PYG{n}{deep}\PYG{o}{=}\PYG{k+kc}{True}\PYG{p}{)}

\PYG{c+c1}{\PYGZsh{} Start a plotter object and set the scalars to the Z height}
\PYG{n}{plobj} \PYG{o}{=} \PYG{n}{vtkInterface}\PYG{o}{.}\PYG{n}{PlotClass}\PYG{p}{(}\PYG{p}{)}
\PYG{n}{plobj}\PYG{o}{.}\PYG{n}{AddMesh}\PYG{p}{(}\PYG{n}{sgrid}\PYG{p}{,} \PYG{n}{scalars}\PYG{o}{=}\PYG{n}{Z}\PYG{o}{.}\PYG{n}{ravel}\PYG{p}{(}\PYG{p}{)}\PYG{p}{)}
\PYG{n}{plobj}\PYG{o}{.}\PYG{n}{Plot}\PYG{p}{(}\PYG{n}{autoclose}\PYG{o}{=}\PYG{k+kc}{False}\PYG{p}{)}

\PYG{c+c1}{\PYGZsh{} Open a gif}
\PYG{n}{plobj}\PYG{o}{.}\PYG{n}{OpenGif}\PYG{p}{(}\PYG{l+s+s1}{\PYGZsq{}}\PYG{l+s+s1}{wave.gif}\PYG{l+s+s1}{\PYGZsq{}}\PYG{p}{)}

\PYG{c+c1}{\PYGZsh{} Update Z and write a frame for each updated position}
\PYG{n}{nframe} \PYG{o}{=} \PYG{l+m+mi}{15}
\PYG{k}{for} \PYG{n}{phase} \PYG{o+ow}{in} \PYG{n}{np}\PYG{o}{.}\PYG{n}{linspace}\PYG{p}{(}\PYG{l+m+mi}{0}\PYG{p}{,} \PYG{l+m+mi}{2}\PYG{o}{*}\PYG{n}{np}\PYG{o}{.}\PYG{n}{pi}\PYG{p}{,} \PYG{n}{nframe} \PYG{o}{+} \PYG{l+m+mi}{1}\PYG{p}{)}\PYG{p}{[}\PYG{p}{:}\PYG{n}{nframe}\PYG{p}{]}\PYG{p}{:}
    \PYG{n}{Z} \PYG{o}{=} \PYG{n}{np}\PYG{o}{.}\PYG{n}{sin}\PYG{p}{(}\PYG{n}{R} \PYG{o}{+} \PYG{n}{phase}\PYG{p}{)}
    \PYG{n}{pts}\PYG{p}{[}\PYG{p}{:}\PYG{p}{,} \PYG{o}{\PYGZhy{}}\PYG{l+m+mi}{1}\PYG{p}{]} \PYG{o}{=} \PYG{n}{Z}\PYG{o}{.}\PYG{n}{ravel}\PYG{p}{(}\PYG{p}{)}
    \PYG{n}{plobj}\PYG{o}{.}\PYG{n}{UpdateCoordinates}\PYG{p}{(}\PYG{n}{pts}\PYG{p}{)}
    \PYG{n}{plobj}\PYG{o}{.}\PYG{n}{UpdateScalars}\PYG{p}{(}\PYG{n}{Z}\PYG{o}{.}\PYG{n}{ravel}\PYG{p}{(}\PYG{p}{)}\PYG{p}{)}

    \PYG{n}{plobj}\PYG{o}{.}\PYG{n}{WriteFrame}\PYG{p}{(}\PYG{p}{)}

\PYG{c+c1}{\PYGZsh{} Close movie and delete object}
\PYG{n}{plobj}\PYG{o}{.}\PYG{n}{Close}\PYG{p}{(}\PYG{p}{)}
\PYG{k}{del} \PYG{n}{plobj}
\end{sphinxVerbatim}

\noindent\sphinxincludegraphics{{wave}.gif}


\chapter{Contents}
\label{\detokenize{index:contents}}

\section{Installation}
\label{\detokenize{installation:installation}}\label{\detokenize{installation:install-ref}}\label{\detokenize{installation::doc}}
Installing vtkInterface is quite simple, but first you will need VTK.
VTK can be easily compiled on a UNIX or Linux machine, but with a Windows
enviornment it’s easier to use a distribution such as
\sphinxhref{https://www.continuum.io/downloads}{Anaconda}.

The installation directions are different depending on your OS.  See below


\subsection{Windows Installation}
\label{\detokenize{installation:windows-installation}}

\subsubsection{Install VTK}
\label{\detokenize{installation:install-vtk}}
Install VTK by installing from a distribution like \sphinxhref{https://www.continuum.io/downloads}{Anaconda} and then installing VTK for Python 3.6 by running the following from a command prompt:

\begin{sphinxVerbatim}[commandchars=\\\{\}]
\PYG{n}{conda} \PYG{n}{install} \PYG{o}{\PYGZhy{}}\PYG{n}{c} \PYG{n}{clinicalgraphics} \PYG{n}{vtk}\PYG{o}{=}\PYG{l+m+mf}{7.1}\PYG{o}{.}\PYG{l+m+mi}{0}
\end{sphinxVerbatim}

Or, if you’re using Python 2.7, install using:

\begin{sphinxVerbatim}[commandchars=\\\{\}]
\PYG{n}{conda} \PYG{n}{install} \PYG{o}{\PYGZhy{}}\PYG{n}{c} \PYG{n}{anaconda} \PYG{n}{vtk}\PYG{o}{=}\PYG{l+m+mf}{6.3}\PYG{o}{.}\PYG{l+m+mi}{0}
\end{sphinxVerbatim}

You can also install vtk from the source by following these \sphinxhref{http://www.vtk.org/Wiki/VTK/Building/Windows}{Directions}.  This is generally quite difficult in Windows.


\subsubsection{Install vtkInterface}
\label{\detokenize{installation:install-vtkinterface}}
Install vtkInterface from PyPi \sphinxhref{http://pypi.python.org/pypi/vtkInterface}{PyPi} by running:

\begin{sphinxVerbatim}[commandchars=\\\{\}]
\PYG{n}{pip} \PYG{n}{install} \PYG{n}{vtkInterface}
\end{sphinxVerbatim}

Alternatively, you can install the latest version from GitHub by visiting \sphinxhref{https://github.com/akaszynski/vtkInterface}{vtkInterface}, downloading the source, and running:

\begin{sphinxVerbatim}[commandchars=\\\{\}]
\PYG{n}{cd} \PYG{n}{C}\PYG{p}{:}\PYGZbs{}\PYG{n}{Where}\PYGZbs{}\PYG{n}{You}\PYGZbs{}\PYG{n}{Downloaded}\PYGZbs{}\PYG{n}{vtkInterface}
\PYG{n}{pip} \PYG{n}{install} \PYG{o}{.}
\end{sphinxVerbatim}


\subsection{Linux Installation}
\label{\detokenize{installation:linux-installation}}

\subsubsection{Install VTK}
\label{\detokenize{installation:id2}}
Building VTK from the source under Linux is straightforward.  See \sphinxhref{http://www.vtk.org/Wiki/VTK/Building/Linux}{Building VTK for Linux} and make sure to enable building with Python.  See \sphinxhref{http://www.vtk.org/Wiki/VTK/Tutorials/PythonEnvironmentSetup}{Python Environment Setup} if you have problems loading VTK from Python.

If you have Ubuntu 14.04 or newer, you can also install VTK using the apt
package manager:

\begin{sphinxVerbatim}[commandchars=\\\{\}]
\PYG{n}{sudo} \PYG{n}{apt}\PYG{o}{\PYGZhy{}}\PYG{n}{get} \PYG{n}{install} \PYG{n}{python}\PYG{o}{\PYGZhy{}}\PYG{n}{vtk}
\end{sphinxVerbatim}

This will be an earlier version of VTK and is not recommended.


\subsubsection{Install vtkInterface}
\label{\detokenize{installation:id3}}
Install vtkInterface from \sphinxhref{http://pypi.python.org/pypi/vtkInterface}{PyPi} by running:

\begin{sphinxVerbatim}[commandchars=\\\{\}]
\PYG{n}{pip} \PYG{n}{install} \PYG{n}{vtkInterface}
\end{sphinxVerbatim}

You can also install the latest source from
\sphinxhref{https://github.com/akaszynski/vtkInterface}{GitHub} with:

\begin{sphinxVerbatim}[commandchars=\\\{\}]
\PYG{n}{git} \PYG{n}{clone} \PYG{n}{https}\PYG{p}{:}\PYG{o}{/}\PYG{o}{/}\PYG{n}{github}\PYG{o}{.}\PYG{n}{com}\PYG{o}{/}\PYG{n}{akaszynski}\PYG{o}{/}\PYG{n}{vtkInterface}
\PYG{n}{cd} \PYG{n}{vtkInterface}
\PYG{n}{pip} \PYG{n}{install} \PYG{o}{.}
\end{sphinxVerbatim}


\subsection{Test Installation}
\label{\detokenize{installation:test-installation}}
Regardless of your OS, you can test your installation by running an example
from Tests:

\begin{sphinxVerbatim}[commandchars=\\\{\}]
\PYG{k+kn}{from} \PYG{n+nn}{vtkInterface} \PYG{k}{import} \PYG{n}{Tests}
\PYG{n}{Tests}\PYG{o}{.}\PYG{n}{ShowWave}\PYG{p}{(}\PYG{p}{)}
\end{sphinxVerbatim}

See the examples page for more tests you can run.


\section{Mesh Reading and Writing}
\label{\detokenize{examples:examples-ref}}\label{\detokenize{examples:mesh-reading-and-writing}}\label{\detokenize{examples::doc}}
Both binary and ASCII .ply, .stl, and .vtk files can be read using
vtkInterface.  The vtkInterface package contains example meshes and these can
be loaded with:

\begin{sphinxVerbatim}[commandchars=\\\{\}]
\PYG{k+kn}{import} \PYG{n+nn}{vtkInterface}

\PYG{c+c1}{\PYGZsh{} Get the filename from the examples}
\PYG{k+kn}{from} \PYG{n+nn}{vtkInterface} \PYG{k}{import} \PYG{n}{examples}
\PYG{n}{filename} \PYG{o}{=} \PYG{n}{examples}\PYG{o}{.}\PYG{n}{planefile}

\PYG{c+c1}{\PYGZsh{} Load mesh}
\PYG{n}{mesh} \PYG{o}{=} \PYG{n}{vtkInterface}\PYG{o}{.}\PYG{n}{LoadMesh}\PYG{p}{(}\PYG{n}{filename}\PYG{p}{)}
\end{sphinxVerbatim}

This mesh can then be written to a vtk file using:

\begin{sphinxVerbatim}[commandchars=\\\{\}]
\PYG{n}{mesh}\PYG{o}{.}\PYG{n}{WriteMesh}\PYG{p}{(}\PYG{l+s+s1}{\PYGZsq{}}\PYG{l+s+s1}{plane.vtk}\PYG{l+s+s1}{\PYGZsq{}}\PYG{p}{)}
\end{sphinxVerbatim}

These meshes are identical:

\begin{sphinxVerbatim}[commandchars=\\\{\}]
\PYG{n}{mesh\PYGZus{}from\PYGZus{}vtk} \PYG{o}{=} \PYG{n}{vtkInterface}\PYG{o}{.}\PYG{n}{LoadMesh}\PYG{p}{(}\PYG{l+s+s1}{\PYGZsq{}}\PYG{l+s+s1}{plane.vtk}\PYG{l+s+s1}{\PYGZsq{}}\PYG{p}{)}

\PYG{k+kn}{import} \PYG{n+nn}{numpy} \PYG{k}{as} \PYG{n+nn}{np}
\PYG{n}{np}\PYG{o}{.}\PYG{n}{allclose}\PYG{p}{(}\PYG{n}{mesh\PYGZus{}from\PYGZus{}vtk}\PYG{o}{.}\PYG{n}{GetNumpyPoints}\PYG{p}{(}\PYG{p}{)}\PYG{p}{,} \PYG{n}{mesh}\PYG{o}{.}\PYG{n}{GetNumpyPoints}\PYG{p}{(}\PYG{p}{)}\PYG{p}{)}
\end{sphinxVerbatim}


\section{Mesh Manipulation and Plotting}
\label{\detokenize{examples:mesh-manipulation-and-plotting}}
Meshes can be directly manipulated using numpy or with the built-in
translation and rotation routines.  This example loads two meshes and moves,
scales, and copies them:

\begin{sphinxVerbatim}[commandchars=\\\{\}]
\PYG{c+c1}{\PYGZsh{} Load module and examples}
\PYG{k+kn}{import} \PYG{n+nn}{vtkInterface}
\PYG{k+kn}{from} \PYG{n+nn}{vtkInterface} \PYG{k}{import} \PYG{n}{examples}
\PYG{n}{planefile} \PYG{o}{=} \PYG{n}{examples}\PYG{o}{.}\PYG{n}{planefile}
\PYG{n}{antfile} \PYG{o}{=} \PYG{n}{examples}\PYG{o}{.}\PYG{n}{antfile}

\PYG{c+c1}{\PYGZsh{} load and shrink airplane}
\PYG{n}{airplane} \PYG{o}{=} \PYG{n}{vtkInterface}\PYG{o}{.}\PYG{n}{LoadMesh}\PYG{p}{(}\PYG{n}{planefile}\PYG{p}{)}
\PYG{n}{pts} \PYG{o}{=} \PYG{n}{airplane}\PYG{o}{.}\PYG{n}{GetNumpyPoints}\PYG{p}{(}\PYG{p}{)} \PYG{c+c1}{\PYGZsh{} gets pointer to array}
\PYG{n}{pts} \PYG{o}{/}\PYG{o}{=} \PYG{l+m+mi}{10} \PYG{c+c1}{\PYGZsh{} shrink by 10x}

\PYG{c+c1}{\PYGZsh{} rotate and translate ant so it is on the plane}
\PYG{n}{ant} \PYG{o}{=} \PYG{n}{vtkInterface}\PYG{o}{.}\PYG{n}{LoadMesh}\PYG{p}{(}\PYG{n}{antfile}\PYG{p}{)}
\PYG{n}{ant}\PYG{o}{.}\PYG{n}{RotateX}\PYG{p}{(}\PYG{l+m+mi}{90}\PYG{p}{)}
\PYG{n}{ant}\PYG{o}{.}\PYG{n}{Translate}\PYG{p}{(}\PYG{p}{[}\PYG{l+m+mi}{90}\PYG{p}{,} \PYG{l+m+mi}{60}\PYG{p}{,} \PYG{l+m+mi}{15}\PYG{p}{]}\PYG{p}{)}

\PYG{c+c1}{\PYGZsh{} Make a copy and add another ant}
\PYG{n}{ant\PYGZus{}copy} \PYG{o}{=} \PYG{n}{ant}\PYG{o}{.}\PYG{n}{Copy}\PYG{p}{(}\PYG{p}{)}
\PYG{n}{ant\PYGZus{}copy}\PYG{o}{.}\PYG{n}{Translate}\PYG{p}{(}\PYG{p}{[}\PYG{l+m+mi}{30}\PYG{p}{,} \PYG{l+m+mi}{0}\PYG{p}{,} \PYG{o}{\PYGZhy{}}\PYG{l+m+mi}{10}\PYG{p}{]}\PYG{p}{)}
\end{sphinxVerbatim}

To plot more than one mesh a plotting class must be created to manage the
plotting.  The following code creates the class and plots the meshes with
various colors:

\begin{sphinxVerbatim}[commandchars=\\\{\}]
\PYG{c+c1}{\PYGZsh{} Create plotting object}
\PYG{n}{plobj} \PYG{o}{=} \PYG{n}{vtkInterface}\PYG{o}{.}\PYG{n}{PlotClass}\PYG{p}{(}\PYG{p}{)}
\PYG{n}{plobj}\PYG{o}{.}\PYG{n}{AddMesh}\PYG{p}{(}\PYG{n}{ant}\PYG{p}{,} \PYG{l+s+s1}{\PYGZsq{}}\PYG{l+s+s1}{r}\PYG{l+s+s1}{\PYGZsq{}}\PYG{p}{)}
\PYG{n}{plobj}\PYG{o}{.}\PYG{n}{AddMesh}\PYG{p}{(}\PYG{n}{ant\PYGZus{}copy}\PYG{p}{,} \PYG{l+s+s1}{\PYGZsq{}}\PYG{l+s+s1}{b}\PYG{l+s+s1}{\PYGZsq{}}\PYG{p}{)}

\PYG{c+c1}{\PYGZsh{} Add airplane mesh and make the color equal to the Y position.  Add a}
\PYG{c+c1}{\PYGZsh{} scalar bar associated with this mesh}
\PYG{n}{plane\PYGZus{}scalars} \PYG{o}{=} \PYG{n}{pts}\PYG{p}{[}\PYG{p}{:}\PYG{p}{,} \PYG{l+m+mi}{1}\PYG{p}{]}
\PYG{n}{plobj}\PYG{o}{.}\PYG{n}{AddMesh}\PYG{p}{(}\PYG{n}{airplane}\PYG{p}{,} \PYG{n}{scalars}\PYG{o}{=}\PYG{n}{plane\PYGZus{}scalars}\PYG{p}{,} \PYG{n}{stitle}\PYG{o}{=}\PYG{l+s+s1}{\PYGZsq{}}\PYG{l+s+s1}{Plane Y}\PYG{l+s+se}{\PYGZbs{}n}\PYG{l+s+s1}{Location}\PYG{l+s+s1}{\PYGZsq{}}\PYG{p}{)}

\PYG{c+c1}{\PYGZsh{} Add annotation text}
\PYG{n}{plobj}\PYG{o}{.}\PYG{n}{AddText}\PYG{p}{(}\PYG{l+s+s1}{\PYGZsq{}}\PYG{l+s+s1}{Ants and Plane Example}\PYG{l+s+s1}{\PYGZsq{}}\PYG{p}{)}
\PYG{n}{plobj}\PYG{o}{.}\PYG{n}{Plot}\PYG{p}{(}\PYG{p}{)}

\PYG{c+c1}{\PYGZsh{} Close plotting object}
\PYG{k}{del} \PYG{n}{plobj}
\end{sphinxVerbatim}

\noindent\sphinxincludegraphics{{AntsAndPlane}.png}


\section{Unstructured Grid Plotting}
\label{\detokenize{examples:unstructured-grid-plotting}}
This example shows how you can load an unstructured grid from a vtk file and
create a plot and gif movie by updating the plotting object:

\begin{sphinxVerbatim}[commandchars=\\\{\}]
\PYG{c+c1}{\PYGZsh{} Load module and example file}
\PYG{k+kn}{import} \PYG{n+nn}{vtkInterface}
\PYG{k+kn}{from} \PYG{n+nn}{vtkInterface} \PYG{k}{import} \PYG{n}{examples}
\PYG{k+kn}{import} \PYG{n+nn}{numpy} \PYG{k}{as} \PYG{n+nn}{np}
\PYG{k+kn}{import} \PYG{n+nn}{time}

\PYG{n}{hexfile} \PYG{o}{=} \PYG{n}{examples}\PYG{o}{.}\PYG{n}{hexbeamfile}

\PYG{c+c1}{\PYGZsh{} Load Grid}
\PYG{n}{grid} \PYG{o}{=} \PYG{n}{vtkInterface}\PYG{o}{.}\PYG{n}{LoadGrid}\PYG{p}{(}\PYG{n}{hexfile}\PYG{p}{)}

\PYG{c+c1}{\PYGZsh{} Create fiticious displacements as a function of Z location}
\PYG{n}{pts} \PYG{o}{=} \PYG{n}{grid}\PYG{o}{.}\PYG{n}{GetNumpyPoints}\PYG{p}{(}\PYG{n}{deep}\PYG{o}{=}\PYG{k+kc}{True}\PYG{p}{)}
\PYG{n}{d} \PYG{o}{=} \PYG{n}{np}\PYG{o}{.}\PYG{n}{zeros\PYGZus{}like}\PYG{p}{(}\PYG{n}{pts}\PYG{p}{)}
\PYG{n}{d}\PYG{p}{[}\PYG{p}{:}\PYG{p}{,} \PYG{l+m+mi}{1}\PYG{p}{]} \PYG{o}{=} \PYG{n}{pts}\PYG{p}{[}\PYG{p}{:}\PYG{p}{,} \PYG{l+m+mi}{2}\PYG{p}{]}\PYG{o}{*}\PYG{o}{*}\PYG{l+m+mi}{3}\PYG{o}{/}\PYG{l+m+mi}{250}

\PYG{c+c1}{\PYGZsh{} Displace original grid}
\PYG{n}{grid}\PYG{o}{.}\PYG{n}{SetNumpyPoints}\PYG{p}{(}\PYG{n}{pts} \PYG{o}{+} \PYG{n}{d}\PYG{p}{)}
\end{sphinxVerbatim}

A simple plot can be created by using:

\begin{sphinxVerbatim}[commandchars=\\\{\}]
\PYG{n}{grid}\PYG{o}{.}\PYG{n}{Plot}\PYG{p}{(}\PYG{n}{scalars}\PYG{o}{=}\PYG{n}{d}\PYG{p}{[}\PYG{p}{:}\PYG{p}{,} \PYG{l+m+mi}{1}\PYG{p}{]}\PYG{p}{,} \PYG{n}{stitle}\PYG{o}{=}\PYG{l+s+s1}{\PYGZsq{}}\PYG{l+s+s1}{Y Displacement}\PYG{l+s+s1}{\PYGZsq{}}\PYG{p}{)}
\end{sphinxVerbatim}

A more complex plot can be created using:

\begin{sphinxVerbatim}[commandchars=\\\{\}]
\PYG{c+c1}{\PYGZsh{} Store Camera position.  This can be obtained manually by getting the}
\PYG{c+c1}{\PYGZsh{} output of plobj.Plot()}
\PYG{n}{cpos} \PYG{o}{=} \PYG{p}{[}\PYG{p}{(}\PYG{l+m+mf}{11.915126303095157}\PYG{p}{,} \PYG{l+m+mf}{6.11392754955802}\PYG{p}{,} \PYG{l+m+mf}{3.6124956735471914}\PYG{p}{)}\PYG{p}{,}
         \PYG{p}{(}\PYG{l+m+mf}{0.0}\PYG{p}{,} \PYG{l+m+mf}{0.375}\PYG{p}{,} \PYG{l+m+mf}{2.0}\PYG{p}{)}\PYG{p}{,}
         \PYG{p}{(}\PYG{o}{\PYGZhy{}}\PYG{l+m+mf}{0.42546442225230097}\PYG{p}{,} \PYG{l+m+mf}{0.9024244135964158}\PYG{p}{,} \PYG{o}{\PYGZhy{}}\PYG{l+m+mf}{0.06789847673314177}\PYG{p}{)}\PYG{p}{]}

\PYG{c+c1}{\PYGZsh{} plot this displaced beam}
\PYG{n}{plobj} \PYG{o}{=} \PYG{n}{vtkInterface}\PYG{o}{.}\PYG{n}{PlotClass}\PYG{p}{(}\PYG{p}{)}
\PYG{n}{plobj}\PYG{o}{.}\PYG{n}{AddMesh}\PYG{p}{(}\PYG{n}{grid}\PYG{p}{,} \PYG{n}{scalars}\PYG{o}{=}\PYG{n}{d}\PYG{p}{[}\PYG{p}{:}\PYG{p}{,} \PYG{l+m+mi}{1}\PYG{p}{]}\PYG{p}{,} \PYG{n}{stitle}\PYG{o}{=}\PYG{l+s+s1}{\PYGZsq{}}\PYG{l+s+s1}{Y Displacement}\PYG{l+s+s1}{\PYGZsq{}}\PYG{p}{,}
              \PYG{n}{rng}\PYG{o}{=}\PYG{p}{[}\PYG{o}{\PYGZhy{}}\PYG{n}{d}\PYG{o}{.}\PYG{n}{max}\PYG{p}{(}\PYG{p}{)}\PYG{p}{,} \PYG{n}{d}\PYG{o}{.}\PYG{n}{max}\PYG{p}{(}\PYG{p}{)}\PYG{p}{]}\PYG{p}{)}
\PYG{n}{plobj}\PYG{o}{.}\PYG{n}{AddAxes}\PYG{p}{(}\PYG{p}{)}
\PYG{n}{plobj}\PYG{o}{.}\PYG{n}{SetCameraPosition}\PYG{p}{(}\PYG{n}{cpos}\PYG{p}{)}

\PYG{c+c1}{\PYGZsh{} Don\PYGZsq{}t close so we can take a screenshot}
\PYG{n}{cpos} \PYG{o}{=} \PYG{n}{plobj}\PYG{o}{.}\PYG{n}{Plot}\PYG{p}{(}\PYG{n}{autoclose}\PYG{o}{=}\PYG{k+kc}{False}\PYG{p}{)}
\PYG{n}{plobj}\PYG{o}{.}\PYG{n}{TakeScreenShot}\PYG{p}{(}\PYG{l+s+s1}{\PYGZsq{}}\PYG{l+s+s1}{beam.png}\PYG{l+s+s1}{\PYGZsq{}}\PYG{p}{)}
\PYG{k}{del} \PYG{n}{plobj}
\end{sphinxVerbatim}

\noindent\sphinxincludegraphics{{beam}.png}

You can animiate the motion of the beam by updating the positions and scalars
of the grid copied to the plotting object.  First you have to setup the
plotting object::

\begin{sphinxVerbatim}[commandchars=\\\{\}]
\PYG{c+c1}{\PYGZsh{} Animate plot}
\PYG{n}{plobj} \PYG{o}{=} \PYG{n}{vtkInterface}\PYG{o}{.}\PYG{n}{PlotClass}\PYG{p}{(}\PYG{p}{)}
\PYG{n}{plobj}\PYG{o}{.}\PYG{n}{AddMesh}\PYG{p}{(}\PYG{n}{grid}\PYG{p}{,} \PYG{n}{scalars}\PYG{o}{=}\PYG{n}{d}\PYG{p}{[}\PYG{p}{:}\PYG{p}{,} \PYG{l+m+mi}{1}\PYG{p}{]}\PYG{p}{,} \PYG{n}{stitle}\PYG{o}{=}\PYG{l+s+s1}{\PYGZsq{}}\PYG{l+s+s1}{Y Displacement}\PYG{l+s+s1}{\PYGZsq{}}\PYG{p}{,}
              \PYG{n}{showedges}\PYG{o}{=}\PYG{k+kc}{True}\PYG{p}{,} \PYG{n}{rng}\PYG{o}{=}\PYG{p}{[}\PYG{o}{\PYGZhy{}}\PYG{n}{d}\PYG{o}{.}\PYG{n}{max}\PYG{p}{(}\PYG{p}{)}\PYG{p}{,} \PYG{n}{d}\PYG{o}{.}\PYG{n}{max}\PYG{p}{(}\PYG{p}{)}\PYG{p}{]}\PYG{p}{,}
              \PYG{n}{interpolatebeforemap}\PYG{o}{=}\PYG{k+kc}{True}\PYG{p}{)}
\PYG{n}{plobj}\PYG{o}{.}\PYG{n}{AddAxes}\PYG{p}{(}\PYG{p}{)}
\PYG{n}{plobj}\PYG{o}{.}\PYG{n}{SetCameraPosition}\PYG{p}{(}\PYG{n}{cpos}\PYG{p}{)}
\end{sphinxVerbatim}

You then open the render window by plotting befroe before opening movie file.
Set autoclose to False so
the plobj doesn’t close automatically.  Disabling interactive means
the plot will automatically continue without waiting for the user to
exit the window:

\begin{sphinxVerbatim}[commandchars=\\\{\}]
\PYG{n}{plobj}\PYG{o}{.}\PYG{n}{Plot}\PYG{p}{(}\PYG{n}{interactive}\PYG{o}{=}\PYG{k+kc}{False}\PYG{p}{,} \PYG{n}{autoclose}\PYG{o}{=}\PYG{k+kc}{False}\PYG{p}{,} \PYG{n}{window\PYGZus{}size}\PYG{o}{=}\PYG{p}{[}\PYG{l+m+mi}{800}\PYG{p}{,} \PYG{l+m+mi}{600}\PYG{p}{]}\PYG{p}{)}

\PYG{c+c1}{\PYGZsh{} open movie file.  A mp4 file can be written instead.  Requires moviepy}
\PYG{c+c1}{\PYGZsh{}plobj.OpenMovie(\PYGZsq{}beam.mp4\PYGZsq{})}
\PYG{n}{plobj}\PYG{o}{.}\PYG{n}{OpenGif}\PYG{p}{(}\PYG{l+s+s1}{\PYGZsq{}}\PYG{l+s+s1}{beam.gif}\PYG{l+s+s1}{\PYGZsq{}}\PYG{p}{)}

\PYG{c+c1}{\PYGZsh{} Modify position of the beam cyclically}
\PYG{k}{for} \PYG{n}{phase} \PYG{o+ow}{in} \PYG{n}{np}\PYG{o}{.}\PYG{n}{linspace}\PYG{p}{(}\PYG{l+m+mi}{0}\PYG{p}{,} \PYG{l+m+mi}{2}\PYG{o}{*}\PYG{n}{np}\PYG{o}{.}\PYG{n}{pi}\PYG{p}{,} \PYG{l+m+mi}{20}\PYG{p}{)}\PYG{p}{:}
    \PYG{n}{plobj}\PYG{o}{.}\PYG{n}{UpdateCoordinates}\PYG{p}{(}\PYG{n}{pts} \PYG{o}{+} \PYG{n}{d}\PYG{o}{*}\PYG{n}{np}\PYG{o}{.}\PYG{n}{cos}\PYG{p}{(}\PYG{n}{phase}\PYG{p}{)}\PYG{p}{,} \PYG{n}{render}\PYG{o}{=}\PYG{k+kc}{False}\PYG{p}{)}
    \PYG{n}{plobj}\PYG{o}{.}\PYG{n}{UpdateScalars}\PYG{p}{(}\PYG{n}{d}\PYG{p}{[}\PYG{p}{:}\PYG{p}{,} \PYG{l+m+mi}{1}\PYG{p}{]}\PYG{o}{*}\PYG{n}{np}\PYG{o}{.}\PYG{n}{cos}\PYG{p}{(}\PYG{n}{phase}\PYG{p}{)}\PYG{p}{,} \PYG{n}{render}\PYG{o}{=}\PYG{k+kc}{False}\PYG{p}{)}
    \PYG{n}{plobj}\PYG{o}{.}\PYG{n}{Render}\PYG{p}{(}\PYG{p}{)}
    \PYG{n}{plobj}\PYG{o}{.}\PYG{n}{WriteFrame}\PYG{p}{(}\PYG{p}{)}

\PYG{c+c1}{\PYGZsh{} Close the movie}
\PYG{n}{plobj}\PYG{o}{.}\PYG{n}{Close}\PYG{p}{(}\PYG{p}{)}
\PYG{k}{del} \PYG{n}{plobj}
\end{sphinxVerbatim}

\noindent\sphinxincludegraphics{{beam}.gif}

You can also render the beam as as a wireframe object:

\begin{sphinxVerbatim}[commandchars=\\\{\}]
\PYG{c+c1}{\PYGZsh{} Animate plot as a wireframe}
\PYG{n}{plobj} \PYG{o}{=} \PYG{n}{vtkInterface}\PYG{o}{.}\PYG{n}{PlotClass}\PYG{p}{(}\PYG{p}{)}
\PYG{n}{plobj}\PYG{o}{.}\PYG{n}{AddMesh}\PYG{p}{(}\PYG{n}{grid}\PYG{p}{,} \PYG{n}{scalars}\PYG{o}{=}\PYG{n}{d}\PYG{p}{[}\PYG{p}{:}\PYG{p}{,} \PYG{l+m+mi}{1}\PYG{p}{]}\PYG{p}{,} \PYG{n}{stitle}\PYG{o}{=}\PYG{l+s+s1}{\PYGZsq{}}\PYG{l+s+s1}{Y Displacement}\PYG{l+s+s1}{\PYGZsq{}}\PYG{p}{,} \PYG{n}{showedges}\PYG{o}{=}\PYG{k+kc}{True}\PYG{p}{,}
              \PYG{n}{rng}\PYG{o}{=}\PYG{p}{[}\PYG{o}{\PYGZhy{}}\PYG{n}{d}\PYG{o}{.}\PYG{n}{max}\PYG{p}{(}\PYG{p}{)}\PYG{p}{,} \PYG{n}{d}\PYG{o}{.}\PYG{n}{max}\PYG{p}{(}\PYG{p}{)}\PYG{p}{]}\PYG{p}{,} \PYG{n}{interpolatebeforemap}\PYG{o}{=}\PYG{k+kc}{True}\PYG{p}{,}
              \PYG{n}{style}\PYG{o}{=}\PYG{l+s+s1}{\PYGZsq{}}\PYG{l+s+s1}{wireframe}\PYG{l+s+s1}{\PYGZsq{}}\PYG{p}{)}
\PYG{n}{plobj}\PYG{o}{.}\PYG{n}{AddAxes}\PYG{p}{(}\PYG{p}{)}
\PYG{n}{plobj}\PYG{o}{.}\PYG{n}{SetCameraPosition}\PYG{p}{(}\PYG{n}{cpos}\PYG{p}{)}
\PYG{n}{plobj}\PYG{o}{.}\PYG{n}{Plot}\PYG{p}{(}\PYG{n}{interactive}\PYG{o}{=}\PYG{k+kc}{False}\PYG{p}{,} \PYG{n}{autoclose}\PYG{o}{=}\PYG{k+kc}{False}\PYG{p}{,} \PYG{n}{window\PYGZus{}size}\PYG{o}{=}\PYG{p}{[}\PYG{l+m+mi}{800}\PYG{p}{,} \PYG{l+m+mi}{600}\PYG{p}{]}\PYG{p}{)}

\PYG{c+c1}{\PYGZsh{}plobj.OpenMovie(\PYGZsq{}beam.mp4\PYGZsq{})}
\PYG{n}{plobj}\PYG{o}{.}\PYG{n}{OpenGif}\PYG{p}{(}\PYG{l+s+s1}{\PYGZsq{}}\PYG{l+s+s1}{beam\PYGZus{}wireframe.gif}\PYG{l+s+s1}{\PYGZsq{}}\PYG{p}{)}
\PYG{k}{for} \PYG{n}{phase} \PYG{o+ow}{in} \PYG{n}{np}\PYG{o}{.}\PYG{n}{linspace}\PYG{p}{(}\PYG{l+m+mi}{0}\PYG{p}{,} \PYG{l+m+mi}{2}\PYG{o}{*}\PYG{n}{np}\PYG{o}{.}\PYG{n}{pi}\PYG{p}{,} \PYG{l+m+mi}{20}\PYG{p}{)}\PYG{p}{:}
    \PYG{n}{plobj}\PYG{o}{.}\PYG{n}{UpdateCoordinates}\PYG{p}{(}\PYG{n}{pts} \PYG{o}{+} \PYG{n}{d}\PYG{o}{*}\PYG{n}{np}\PYG{o}{.}\PYG{n}{cos}\PYG{p}{(}\PYG{n}{phase}\PYG{p}{)}\PYG{p}{,} \PYG{n}{render}\PYG{o}{=}\PYG{k+kc}{False}\PYG{p}{)}
    \PYG{n}{plobj}\PYG{o}{.}\PYG{n}{UpdateScalars}\PYG{p}{(}\PYG{n}{d}\PYG{p}{[}\PYG{p}{:}\PYG{p}{,} \PYG{l+m+mi}{1}\PYG{p}{]}\PYG{o}{*}\PYG{n}{np}\PYG{o}{.}\PYG{n}{cos}\PYG{p}{(}\PYG{n}{phase}\PYG{p}{)}\PYG{p}{,} \PYG{n}{render}\PYG{o}{=}\PYG{k+kc}{False}\PYG{p}{)}
    \PYG{n}{plobj}\PYG{o}{.}\PYG{n}{Render}\PYG{p}{(}\PYG{p}{)}
    \PYG{n}{plobj}\PYG{o}{.}\PYG{n}{WriteFrame}\PYG{p}{(}\PYG{p}{)}
    \PYG{n}{time}\PYG{o}{.}\PYG{n}{sleep}\PYG{p}{(}\PYG{l+m+mf}{0.01}\PYG{p}{)}

\PYG{n}{plobj}\PYG{o}{.}\PYG{n}{Close}\PYG{p}{(}\PYG{p}{)}
\PYG{k}{del} \PYG{n}{plobj}
\end{sphinxVerbatim}

\noindent\sphinxincludegraphics{{beam_wireframe}.gif}


\chapter{Indices and tables}
\label{\detokenize{index:indices-and-tables}}\begin{itemize}
\item {} 
\DUrole{xref,std,std-ref}{genindex}

\item {} 
\DUrole{xref,std,std-ref}{modindex}

\item {} 
\DUrole{xref,std,std-ref}{search}

\end{itemize}



\renewcommand{\indexname}{Index}
\printindex
\end{document}